% This file was converted to LaTeX by Writer2LaTeX ver. 1.4
% see http://writer2latex.sourceforge.net for more info
\documentclass[a4paper]{article}
\usepackage{amsmath,amssymb,amsfonts}
\usepackage{fontspec}
\usepackage{xunicode}
\usepackage{xltxtra}
\usepackage{polyglossia}
\usepackage{ctex}
\usepackage{color}
\usepackage{array}
\usepackage{hhline}
\usepackage{hyperref}
\hypersetup{colorlinks=true, linkcolor=blue, citecolor=blue, filecolor=blue, urlcolor=blue}
% Outline numbering
\setcounter{secnumdepth}{0}
% List styles
\newcommand\liststyleWWNumv{%
\renewcommand\theenumi{\arabic{enumi}}
\renewcommand\theenumii{\arabic{enumi}.\arabic{enumii}}
\renewcommand\theenumiii{\arabic{enumi}.\arabic{enumii}.\arabic{enumiii}}
\renewcommand\theenumiv{\arabic{enumi}.\arabic{enumii}.\arabic{enumiii}.\arabic{enumiv}}
\renewcommand\labelenumi{\theenumi.}
\renewcommand\labelenumii{\theenumii.}
\renewcommand\labelenumiii{\theenumiii.}
\renewcommand\labelenumiv{\theenumiv.}
}
\newcommand\liststyleWWNumviii{%
\renewcommand\theenumi{\arabic{enumi}}
\renewcommand\theenumii{\arabic{enumi}.\arabic{enumii}}
\renewcommand\theenumiii{\arabic{enumi}.\arabic{enumii}.\arabic{enumiii}}
\renewcommand\theenumiv{\arabic{enumi}.\arabic{enumii}.\arabic{enumiii}.\arabic{enumiv}}
\renewcommand\labelenumi{\theenumi.}
\renewcommand\labelenumii{\theenumii.}
\renewcommand\labelenumiii{\theenumiii.}
\renewcommand\labelenumiv{\theenumiv.}
}
\newcommand\liststyleWWNumiv{%
\renewcommand\theenumi{\arabic{enumi}}
\renewcommand\theenumii{\arabic{enumi}.\arabic{enumii}}
\renewcommand\theenumiii{\arabic{enumi}.\arabic{enumii}.\arabic{enumiii}}
\renewcommand\theenumiv{\arabic{enumi}.\arabic{enumii}.\arabic{enumiii}.\arabic{enumiv}}
\renewcommand\labelenumi{\theenumi.}
\renewcommand\labelenumii{\theenumii.}
\renewcommand\labelenumiii{\theenumiii.}
\renewcommand\labelenumiv{\theenumiv.}
}
\newcommand\liststyleWWNumix{%
\renewcommand\theenumi{\arabic{enumi}}
\renewcommand\theenumii{\arabic{enumi}.\arabic{enumii}}
\renewcommand\theenumiii{\arabic{enumi}.\arabic{enumii}.\arabic{enumiii}}
\renewcommand\theenumiv{\arabic{enumi}.\arabic{enumii}.\arabic{enumiii}.\arabic{enumiv}}
\renewcommand\labelenumi{\theenumi.}
\renewcommand\labelenumii{\theenumii.}
\renewcommand\labelenumiii{\theenumiii.}
\renewcommand\labelenumiv{\theenumiv.}
}
\newcommand\liststyleWWNumvi{%
\renewcommand\theenumi{\arabic{enumi}}
\renewcommand\theenumii{\arabic{enumi}.\arabic{enumii}}
\renewcommand\theenumiii{\arabic{enumi}.\arabic{enumii}.\arabic{enumiii}}
\renewcommand\theenumiv{\arabic{enumi}.\arabic{enumii}.\arabic{enumiii}.\arabic{enumiv}}
\renewcommand\labelenumi{\theenumi.}
\renewcommand\labelenumii{\theenumii.}
\renewcommand\labelenumiii{\theenumiii.}
\renewcommand\labelenumiv{\theenumiv.}
}
\newcommand\liststyleWWNumvii{%
\renewcommand\theenumi{\arabic{enumi}}
\renewcommand\theenumii{\arabic{enumi}.\arabic{enumii}}
\renewcommand\theenumiii{\arabic{enumi}.\arabic{enumii}.\arabic{enumiii}}
\renewcommand\theenumiv{\arabic{enumi}.\arabic{enumii}.\arabic{enumiii}.\arabic{enumiv}}
\renewcommand\labelenumi{\theenumi.}
\renewcommand\labelenumii{\theenumii.}
\renewcommand\labelenumiii{\theenumiii.}
\renewcommand\labelenumiv{\theenumiv.}
}
\newcommand\liststyleWWNumiii{%
\renewcommand\theenumi{\arabic{enumi}}
\renewcommand\theenumii{\arabic{enumi}.\arabic{enumii}}
\renewcommand\theenumiii{\arabic{enumi}.\arabic{enumii}.\arabic{enumiii}}
\renewcommand\theenumiv{\arabic{enumi}.\arabic{enumii}.\arabic{enumiii}.\arabic{enumiv}}
\renewcommand\labelenumi{\theenumi.}
\renewcommand\labelenumii{\theenumii.}
\renewcommand\labelenumiii{\theenumiii.}
\renewcommand\labelenumiv{\theenumiv.}
}
\newcommand\liststyleWWNumi{%
\renewcommand\theenumi{\arabic{enumi}}
\renewcommand\theenumii{\arabic{enumi}.\arabic{enumii}}
\renewcommand\theenumiii{\arabic{enumi}.\arabic{enumii}.\arabic{enumiii}}
\renewcommand\theenumiv{\arabic{enumi}.\arabic{enumii}.\arabic{enumiii}.\arabic{enumiv}}
\renewcommand\labelenumi{\theenumi.}
\renewcommand\labelenumii{\theenumii.}
\renewcommand\labelenumiii{\theenumiii.}
\renewcommand\labelenumiv{\theenumiv.}
}
\newcommand\liststyleWWNumii{%
\renewcommand\theenumi{\arabic{enumi}}
\renewcommand\theenumii{\arabic{enumi}.\arabic{enumii}}
\renewcommand\theenumiii{\arabic{enumi}.\arabic{enumii}.\arabic{enumiii}}
\renewcommand\theenumiv{\arabic{enumi}.\arabic{enumii}.\arabic{enumiii}.\arabic{enumiv}}
\renewcommand\labelenumi{\theenumi.}
\renewcommand\labelenumii{\theenumii.}
\renewcommand\labelenumiii{\theenumiii.}
\renewcommand\labelenumiv{\theenumiv.}
}
% Page layout (geometry)
\setlength\voffset{-1in}
\setlength\hoffset{-1in}
\setlength\topmargin{2.54cm}
\setlength\oddsidemargin{2.54cm}
\setlength\textheight{23.451cm}
\setlength\textwidth{15.920999cm}
\setlength\footskip{2.4390001cm}
\setlength\headheight{0cm}
\setlength\headsep{0cm}
% Footnote rule
\setlength{\skip\footins}{0.119cm}
\renewcommand\footnoterule{\vspace*{-0.018cm}\setlength\leftskip{0pt}\setlength\rightskip{0pt plus 1fil}\noindent\textcolor{black}{\rule{0.25\columnwidth}{0.018cm}}\vspace*{0.101cm}}
% Pages styles
\makeatletter
\newcommand\ps@Standard{
  \renewcommand\@oddhead{}
  \renewcommand\@evenhead{}
  \renewcommand\@oddfoot{\textcolor[rgb]{0.4,0.4,0.4}{- \thepage{}\ -}}
  \renewcommand\@evenfoot{\@oddfoot}
  \renewcommand\thepage{\arabic{page}}
}
\makeatother
\pagestyle{Standard}
\title{}
\author{Charles Yang}
\date{2018-03-28}
\begin{document}
\clearpage\setcounter{page}{1}\pagestyle{Standard}
{\centering
	{\Large Ingress社团总章程}
\par}


{\centering\color[rgb]{0.4,0.4,0.4}
第四版
\par}

\section[第一章 总则]{第一章 总则}
\liststyleWWNumv
\begin{enumerate}
\item 社团名称:电子科技大学Ingress社团。

\begin{enumerate}
\item 非正式地,亦可称为Ingress协会或Ingress社。
\end{enumerate}
\item 社团宗旨: 淬炼身心,追寻旅程。
\item 社团原则:遵守宪法、法律、法规,校团委和社团联合会的规定,遵守社会道德风尚。
\item 社团职能及活动:使更多的同学了解Ingress,通过社团这个集体来增强同学们的凝聚力和社会能力,通过探索与发现身边事物的方式使同学们学到更多课外知识,同时在快乐中提高身体素质和社会素养。
\item 社员活动的准则:言行高尚、谦卑有礼、尊重隐私、遵守规则。
\end{enumerate}

\bigskip

\section[第二章 组织机构及管理]{第二章 组织机构及管理}
\liststyleWWNumviii
\begin{enumerate}
\item 社团组织形式为各个会员拥有充分自主权利的社群模式。
\item 社团设立社长一人(必要),社群负责人两人(必要),活动管理与策划负责人两人,宣传负责人两人,非必要负责人可随时由社长任命或取消任命。

\begin{enumerate}
\item 社长由社团会员中本校学生担任,采用社群推荐的方式产生以及进行换届,其职能包括:

\begin{enumerate}
\item 向成员汇报在职期间的社团总结,并据此决定社团的发展计划。
\item 协商及公布社团章程,任免各个规定可任免的负责人,授予社团的特殊称号和奖励。
\item 安排以及协助社团的具体工作。
\item 审计社团的财务计划,审核各项开支。
\item 对外代表整个社团,在授权状况下,指定代表与其它组织签订协议。
\item 维持社团的秩序,尤其是竞争性活动。
\end{enumerate}
\item 社群负责人由社群内部自行产生,不限定产生方式,依照Ingress两大社群(通常称作ENL及RES),两社群各选出一名,职能包括:

\begin{enumerate}
\item 在所属社群发起和组织社团名义的活动。
\item 提交活动的预算申请和报销申请。
\item 对外宣传社团,对内管理社群内与社团相关的事务。
\item 调解包括但不限于电子科大地区的社群矛盾与冲突。
\end{enumerate}
\item 活动管理与策划负责人,宣传负责人,任务由社长指定。
\end{enumerate}
\end{enumerate}

\bigskip

\section[第三章 会员]{第三章 会员}
\liststyleWWNumiv
\begin{enumerate}
\item 入会资格认定:

\begin{enumerate}
\item 凡是电子科技大学注册本科生、硕士或者博士研究生,并能遵守社团章程、按时交纳会费,即可申请成为社团会员。
\item 除非通过社长同意,社团会员须在进入社团三天及以内出示自己的Codename(即ID)以验证身份。
\item 任何无不良记录的校外有意参加者,如能遵守社团章程并且申请为所在社群负责人通过,即可成为社团校外成员,但在会费和管理上略有差异,详见本章其他内容。
\end{enumerate}
\item 入会及退会程序:

\begin{enumerate}
\item 申请人通过社团招新或提出申请,通过登记即可注册成为会员。
\item 自愿退会:应由会员本人提出请求,通过身份验证程序确认为本人,并通知全体会员。
\item 中止会员资格:严重违反社团章程者,可由所在社群负责人或社长讨论决定并以适当方式予以宣布,取消会员资格。
\end{enumerate}
\item 会员的义务

\begin{enumerate}
\item 必要情况下,按照相关规定缴纳会费。
\item 维护社团的合法权益和声誉。
\item 对社团工作进行批评建议和监督。
\item 执行社团决议。
\item 维护社团内部稳定。
\end{enumerate}
\item 会员权利

\begin{enumerate}
\item 提出任何合理的报销申请的权利,但要先提交社群负责人汇总。
\item 有权在不违反章程的情况下以任意方式表达对社团任何形式的意见以及建议。
\item 知情权,即获得社团活动或公告通知的主动推送。
\item 拥有自由退会权,允许在任何情况下自由退出社团。
\item 获得社团服务的优先权。
\end{enumerate}
\item 校外会员管理

\begin{enumerate}
\item 校外社员有权提出报销申请及参加校内各项社团活动。
\item 在社团组织的活动中,校外会员的自发行为,本社团概不负责。
\end{enumerate}
\end{enumerate}

\bigskip

\section[第四章 社群与社长]{第四章 社群与社长}
\liststyleWWNumix
\begin{enumerate}
\item 社群的定义及范围

\begin{enumerate}
\item 根据Ingress社区的公认方式,按照Ingress中的对立团队,分为两个社群。
\item 社群作为一个群体,成员不一定是社团成员,社团对社群亦不存在直接管辖权。
\end{enumerate}
\item 社群的责任免除

\begin{enumerate}
\item 未由社团通过或认可的活动,经费一律不予报销,社团对此不存在管理责任。
\item 社群的组织形式,自发的活动,与社团的管理和计划无任何关联。
\end{enumerate}
\item 社群的集体职权

\begin{enumerate}
\item 作为社群代表及社团管理层,社群负责人有权根据社群当前状况,提出合理的活动和经费申请。
\item 非社团的社群活动有权请求社群负责人给予经费支持,但需由社长审核。
\end{enumerate}
\item 社长的担任,连任与辞职

\begin{enumerate}
\item 社长由社团成员自愿提出申请,通过社群范围内投票获得最多且至少三分之一的票数,进而担任社长。
\item 社长原则上不得连任,但若无任何有效提名,允许连任次数不超过一次。
\item 校外社团成员不得担任社长。
\item 社长有权辞职,社长辞职或社长无法产生的情况下,由前任社长指定一位临时负责人处理相关事务,直至选出下一任社长。
\end{enumerate}
\end{enumerate}

\bigskip

\section[第五章 职权界定]{第五章 职权界定}
\liststyleWWNumvi
\begin{enumerate}
\item 越权的定义及处理

\begin{enumerate}
\item 社团内,任何不遵循社团章程的行为即视为越权。
\item 对于章程上尚未指定的内容,由社群负责人和社长决定。
\item 对于严重违规的个人,取消社员资格。
\item 社群负责人如出现过不真实的经费申请或违反章程中社区准则的内容,无论大小,一律永久取消社群负责人和社长提名,并重新选出社群负责人。
\end{enumerate}
\item 社长的职权界定

\begin{enumerate}
\item 社长不能任免,暂停,撤销社群负责人或其权利。
\item 社长的决策不能存有偏向性。
\item 社长不得干涉下一届候选人和相关换届事宜。
\end{enumerate}
\item 社群负责人职权界定

\begin{enumerate}
\item 社群负责人不得未经授权以社团身份干涉其他地区社群事务。
\item 社群负责人不得挑起矛盾或冲突。
\end{enumerate}
\item 社长任免的负责人职权界定

\begin{enumerate}
\item 应当仅在给予的职权下工作。
\item 具体职权由任免时的通知为准。
\end{enumerate}
\item 社团的责任范围与责任免除

\begin{enumerate}
\item 社团仅作为协调电子科大内社群活动与发展的机构,任何社群的自发行为与社团无关。
\item 本社成员的个人行为,社团不承担相应后果。
\item 社团不干涉社群的自由活动,但社群涉及社团的部分,应由社群负责人负责。
\item 社团将尽力保证每位成员都能正常参与社团及社群活动,但不承担因不可抗力因素导致社团外的活动无法进行的相关后果。
\item 社团的认定信息发布渠道为社团所在学院公众号及网站、Telegram及QQ群,因接受非社团发布的不实消息导致的后果本社团没有承担的义务。
\end{enumerate}
\end{enumerate}

\bigskip

\section[第六章 财务审计与决议]{第六章 财务审计与决议}
\liststyleWWNumvii
\begin{enumerate}
\item 任何会员均有权随时查看社团的开支记录,下列情况视为违规,社员可向社群负责人或者社长反映

\begin{enumerate}
\item 使用不真实凭据获得社团经费的。
\item 活动开支超出预算且带有社群偏向性的。
\item 铺张浪费或无法查明开支的。
\end{enumerate}
\item 社团的经费来源与使用

\begin{enumerate}
\item 一般情况,社团活动的经费由参与社团活动的社团成员承担。
\item 在大型活动或需要较多开支的情况下,按照规定向校社联提出报销申请。
\item 本社团不强制收取社团费用,接受社会各界无偿赞助。
\end{enumerate}
\item 社团的合理开支

\begin{enumerate}
\item 活动经费报销需由社群负责人汇总,将列表及凭据交予社长处。
\item 社长有权否决任何报销申请。
\item 报销经费由社群负责人自行汇总确定。
\item 高于300人民币的报销申请需两位社群负责人同时审核通过。
\item 社员有权随时查看社团的收支情况,并对其进行监督。
\item 社团的管理层有义务实现和维护财务透明化,接受社员的监督。
\end{enumerate}
\item 章程及其修正案

\begin{enumerate}
\item 在两社群负责人和社长协商一致后才能:

\begin{enumerate}
\item 发布或修改社团约束性文件。
\item 建议社团章程修订案草案。
\end{enumerate}
\item 在两社群负责人,社长,以及除去弃权后50\%以上社团成员协商一致后才能

\begin{enumerate}
\item 改变社团名称,宗旨,定位和社团章程中对于职务和权利的约束。
\item 审议并通过社团章程修订案草案。
\end{enumerate}
\item 在两社群负责人,社长,以及除去不多于10\%弃权后75\%以上社团成员协商一致后才能

\begin{enumerate}
\item 终止和停止终止社团。
\end{enumerate}
\end{enumerate}
\end{enumerate}

\bigskip

\section[第七章 社团活动]{第七章 社团活动}
\liststyleWWNumiii
\begin{enumerate}
\item 社团活动的申报

\begin{enumerate}
\item 一般的社团活动,需要向社群负责人提前说明,作为活动经费报销的依据。
\item 以下类型的活动不得向社团申报

\begin{enumerate}
\item 无需申报的活动,包括人数少于5人或无特殊意义的活动,例如聚餐。
\item 不得申报的活动,包含已经获得其他组织报销或违反章程的。
\item 无法提供信息的活动。
\end{enumerate}
\item 以下活动需要社长或两位社群负责人同时通过

\begin{enumerate}
\item 涉及大量经费开支的活动。
\item 规模较大的活动。
\end{enumerate}
\end{enumerate}
\item 社团活动的进行

\begin{enumerate}
\item 社长及相应的社群负责人有权临时终止活动的进行。
\item 社团活动的相关经费需求应当提前报告,原则上只报销活动举行前已提交的报销申请。
\end{enumerate}
\end{enumerate}
\section[第八章 隐私权]{第八章 隐私权}
\liststyleWWNumi
\begin{enumerate}
\item 所有会员在入会程序时,必须收到一封隐私权通知,明确相关事项,并告知可能存在的安全风险和规避措施。
\item 成为会员即认定公开自己的Codename(即ID),但成员可要求尽量避免其出现在社团宣传媒体上。
\item 会员在登记注册时须遵守社联相关规定,社团不会主动向社联索取这些信息,在本社团内,会员有权不披露自己的除Codename外其他身份信息。
\item 社团有义务阻止任何人或组织在未经许可的条件下获得这些个人信息:

\begin{enumerate}
\item 姓名、电话号码、电子邮件地址或实际地址。
\item 基于活动的信息,包括位置和习惯。
\end{enumerate}
\item 社团绝不会,也绝不允许

\begin{enumerate}
\item 未经他人许可的,恶意偷拍或公布其位置信息。
\item 未经他人许可的,发布或泄露有关用户身份的其他信息,包括姓名、电话号码、电子邮件地址或实际地址。
\item 未经他人许可,透露旁观者或其他任何人的信息。
\end{enumerate}
\item 社团的宣传过程中

\begin{enumerate}
\item 发表任何可能涉及个人隐私的内容应当征得本人明确同意。
\item 宣传发布者有义务主动维护和检查信息的隐私性。
\end{enumerate}
\item 政策变更知情权

\begin{enumerate}
\item 在社团涉及个人隐私的规定发生时,我们将尽力通知社团成员相关情况并进行确认。
\end{enumerate}
\item 被遗忘权

\begin{enumerate}
\item 社团成员有权请求社团删除指定的全部或部分涉及个人的任何信息。
\end{enumerate}
\end{enumerate}

\bigskip

\section[参考]{参考}
\liststyleWWNumii
\begin{enumerate}
\item 判断是否违反社群规定时,统一以最新<Ingress 社区准则>为准
\end{enumerate}
\end{document}
